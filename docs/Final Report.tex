\documentclass{article}
\usepackage{a4}
\usepackage{listings}
\usepackage{authblk}

\lstset{
	columns=fullflexible,
	frame=single,
	breaklines=true,
}
\title{Protecting democracy with a trustless blockchain based decentralised election system}
\date{\today}

\makeatletter
\def\@maketitle{%
	\newpage
	\null
	\vskip 2em%
	\begin{center}%
		\let \footnote \thanks
		{\LARGE \@title \par}%
		\vskip 1em%
		{\Large VoteBlocks white paper\par}%
		\vskip 1.5em%
		{\large
			\lineskip .5em%
			\begin{tabular}[t]{c}%
				\@author
			\end{tabular}\par}%
		\vskip 1em%
		{\large \@date}%
	\end{center}%
	\par
	\vskip 1.5em}
\makeatother

\begin{document}
	\maketitle
	\begin{abstract}
		Democracy fades as sophisticated attempts of voterfraud are detected, with some even succeeding. VoteBlock attempts to protect democracy by decentralising the election process to ensure the lack of a single point of failure or control, with the help of a mininmal blockchain built from scratch and custom made for elections. It must be understood that while VoteBlock secures the election, it does not secure the voter registration process essential for authorizing each voter, and does not attempt to either. It is an application of the blockchain process in the real world, and distributes trust among multiple entities to ensure fair and free elections. The public participating in said elections must be vigilant in protecting the peripheral technology required to ensure the smooth functioning of VoteBlocks.
	\end{abstract}
	\pagebreak
	\tableofcontents
	\pagebreak
	\section{Introduction}
	The system is broken down into two major components. The block, and the chain. Each chain is made up a number of blocks, all interlinked with each other. Each block is populated with the transactions, i.e votes in our model.
	
	When votes are cast, they are added to a list of unconfirmed transactions. When the block is to be mined, it will be put through a SHA-256 function multiple times in order to match a certain hash pattern. Once the right hash is formed using the nonce, it is added to the block. 
	
	When combined, these entities form blockchains, which are hosted and co-ordinated by a number of Python Flask servers, to ensure redundancy. This ensures that the election process can continue with the support of a simple plug-and-play system, where nodes and servers can be swapped out while in operation to fix issues or improve performance. This is revolutionary as it provides maximum redundancy while allowing industry-standard operational performance. 
	\section{Structure of A Block}
	\subsection{Components}
	Each block consists of the following elements essential to recording votes and identifying induviduality of each voter.
	\begin{enumerate}
		\item Index - To enumerate each block
		\item Transactions - The given transactions in each block (votes)
		\item Timestamp - The timestamp of each block's creation
		\item Previous Hash - The hash of the previous block to facilitate the blockchain
		\item Nonce - The number only used once. Used to supplement the rest of the data to generate a desired hash pattern
	\end{enumerate}
	\subsection{Hashing}
	SHA-256 hashing is used, where the data of the block is first dumped into a string, and then computed after being unicode encoded.
	\subsection{Code Used}
	\textbf{block.py}
	\begin{lstlisting}[language=Python]
from hashlib import sha256
import json

class Block:
	def __init__(self, index, transactions, timestamp, previous_hash, nonce=0):
			self.index = index
			self.transactions = transactions
			self.timestamp = timestamp
			self.previous_hash = previous_hash
			self.nonce = nonce

	def compute_hash(self):
		block_string = json.dumps(self.__dict__, sort_keys=True)
		return sha256(block_string.encode()).hexdigest()
	\end{lstlisting}
\pagebreak
\section{The Blockchain}
\subsection{Components}
\subsubsection{Basic Structuring}
The Blockchain consists of a two objects. 
\begin{enumerate}
	\item Unconfirmed Transactions - To store a list of unconfirmed transactions.
	\item Chain - To store the chain data.
\end{enumerate}
\paragraph{Difficulty}
The difficulty is a simple integer that determines how hard it is to mine a block. The higher this integer is set, the more difficult it is to mine. 

\subparagraph{Method}
This is accomplished by using the number of leading zeros to the hash. A difficulty of 2 will ensure that there are two leading zeros for each accepted hash.
\subsubsection{Genesis Block}
The genesis block of any block chain is the first mined block of that blockchain. Here, we mine it with a list of empty transactions and null (0) data, and we add it to the chain.
\subsubsection{Last Block Property}
A useful property that is used to retrieve the last block added to the blockchain.
\subsubsection{Add Block Methodology}
It is very simple to add a block to the blockchain. It requires only the block object and the proof of work.
\paragraph{Verification}
includes: 
\begin{itemize}
	\item Checking if the proof is valid.
	\item The previous hash referred in the block and the hash of latest block in the chain match.
\end{itemize}
\subsubsection{Proof of Work}
A simple function that tries different nonce values to satisfy the hash pattern requirements set by our difficulty criteria.
\subsubsection{Add New Transaction}
Appends a transaction to the unconfirmed transaction chain
\subsubsection{Proof Verification}
The proof is verified by using the hash supplied and the sha-256 hash computed on the spot.
\subsubsection{Chain Validity Check}
We basically use the valid proof check on each and every block in the chain to check if the block as a whole is valid. It is this process that gives the strength to the blockchain, because it is relatively easy to verify the validity of the chain, compared to creating a fake valid chain from scratch. 
\subsubsection{Mining}
A new block is creating with the given unconfirmed transactions and the proof of work is computed. The add block method is then used to add the new block to the blockchain, and the unconfirmed transactions / votes are removed
\subsection{Code Used}
\begin{lstlisting}[language=Python]
from block import Block
import time

class Blockchain:
	difficulty = 2
	
	def __init__(self):
		self.unconfirmed_transactions = []
		self.chain = []
	
	def create_genesis_block(self):
		genesis_block = Block(0, [], 0, "0")
		genesis_block.hash = genesis_block.compute_hash()
		self.chain.append(genesis_block)
	
	@property
	def last_block(self):
		return self.chain[-1]
	
	def add_block(self, block, proof):
		previous_hash = self.last_block.hash
		
		if previous_hash != block.previous_hash:
			return False
		
		if not Blockchain.is_valid_proof(block, proof):
			return False
		
		block.hash = proof
		self.chain.append(block)
		return True
	
	@staticmethod
	def proof_of_work(block):
		block.nonce = 0
		computed_hash = block.compute_hash()
		while not computed_hash.startswith('0' * Blockchain.difficulty):
			block.nonce += 1
			computed_hash = block.compute_hash()
		
		return computed_hash
	
	def add_new_transaction(self, transaction):
		self.unconfirmed_transactions.append(transaction)
	
	@classmethod
	def is_valid_proof(cls, block, block_hash):
		return (block_hash.startswith('0' * Blockchain.difficulty) and
		block_hash == block.compute_hash())
	
	@classmethod
	def check_chain_validity(cls, chain):
		result = True
		previous_hash = "0"
	
		for block in chain:
			block_hash = block.hash
			# remove the hash field to recompute the hash again
			# using `compute_hash` method.
			delattr(block, "hash")
		
		if not cls.is_valid_proof(block, block_hash) or 
				previous_hash != block.previous_hash:	
			result = False
			break
		
			block.hash, previous_hash = block_hash, block_hash
		
		return result
	
	def mine(self):
		if not self.unconfirmed_transactions:
			return False
		
		last_block = self.last_block
		
		new_block = Block(index=last_block.index + 1,
			transactions=self.unconfirmed_transactions,
			timestamp=time.time(),
			previous_hash=last_block.hash)
		
		proof = self.proof_of_work(new_block)
		self.add_block(new_block, proof)
		
		self.unconfirmed_transactions = []
		return True
\end{lstlisting}
	
\end{document}
